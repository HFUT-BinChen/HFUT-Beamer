% The following code uses the Theme saved in ./HFUTTheme/
\makeatletter
  \def\beamer@calltheme#1#2#3{%
    \def\beamer@themelist{#2}
    \@for\beamer@themename:=\beamer@themelist\do
    {\usepackage[{#1}]{\beamer@themelocation/#3\beamer@themename}}}

  \def\usefolder#1{
    \def\beamer@themelocation{#1}
  }
  \def\beamer@themelocation{}
\usefolder{./HFUTTheme/}
\usetheme{HFUT}

% Set some lengths
\setbeamersize{text margin left=20pt,text margin right=20pt}
\setbeamertemplate{itemize items}[square] 
%\setbeamertemplate{itemize items}{\color{black}$\bullet$} 
\setbeamerfont{page number in head/foot}{size=\scriptsize}
\setbeamertemplate{footline}[myframe number]
\setbeamerfont{page number in head/foot}{size=\tiny}		% Make the slide numbers smaller

% Set some colors and fonts
\setbeamercolor{alerted text}{fg=HFUTred}
\setbeamerfont{alerted text}{}

% This is to put a box around every slide. Uuseful for printouts
\setbeamertemplate{background}{%
\begin{tikzpicture}[overlay,remember picture]
  \draw [line width=0.05mm]
    ($ (current page.north west) + (0.0cm,-0.0cm) $)
    rectangle
    ($ (current page.south east) + (-0.0cm,0.0cm) $);
\end{tikzpicture}%
}

% Math commands
\newcommand{\mc}[1]{\mathcal{#1}}
\newcommand{\set}[1]{\{#1\}}
\newcommand{\eq}[1]{\begin{align*} #1 \end{align*}}
\newcommand{\tr}[1]{\mathrm{#1}}
\newcommand{\ie}{i.e.,~}
\newcommand{\eg}{e.g.,~}
\newcommand{\cf}{cf.~}
\newcommand{\ms}[1]{\mathds{#1}}
\newcommand{\sinc}[1]{\mathrm{sinc}\left(#1\right)}


\definecolor{darkgreen}{rgb}{0,0.65,0}%