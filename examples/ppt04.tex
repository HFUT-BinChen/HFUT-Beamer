
    \section{总结与展望}
    %添加一个目录
    \frame{
     \frametitle{目录}
     \tableofcontents[current,currentsection,sections={<1-5>}]
     \addtocounter{framenumber}{-1}  %目录页不计算页码
    }

    \frame{
      \frametitle{\secname}
      \vspace{-0.5em}
      \begin{block}{总结}
        \footnotesize
        \begin{enumerate}[label={\arabic*}]
          \item 提出了一种构造紧致凸包围多面体--$k$-CBP~的算法;
          \item 构造~$k$-CBP~速度上比现有算法快~3$\sim$8~倍;
          \item 构造的~$k$-CBP~紧致程度比现有的$k$-DOP紧致10\% $\sim$ 40\%;
          \item 提出了一种基于~$k$-CBP~的碰撞检测算法,该算法较$k$-DOP树算法初始化时间快8倍以上,静止场景快0.8 $\sim$ 3.2 倍,运动场景快0.8 $\sim$ 5.6 倍。
        \end{enumerate}
      \end{block}
      \vspace{-0.5em}
      \begin{block}{展望}
        \footnotesize
      \begin{enumerate}[label={\arabic*}]
      
   
      
          \item 碰撞检测算法如何摆脱对AABB树的依赖;应用于近似碰撞检测算法;应用于可变形的模型连续碰撞检测,如何快速更新~$k$-CBP~;
          \item 如何将~$k$-CBP~应用于如机器人抓取、路径规划等其他应用领域中。
        \end{enumerate}
      \end{block}
      
      \note{
        总结一下~\ldots PPT
      }
    }

    \section{主要参考文献}
    \frame[t,allowframebreaks]{
      \frametitle{\secname}
    \printbibliography
    }
    
    \section{感谢}
    \frame{
      \frametitle{\secname}
      \begin{block}{致谢}
        \begin{enumerate}[label={\arabic*}]
          \item 导师XX老师的精心指导;
          \item XX老师帮助;
          \item 研究所各个项目的历练;
          \item XX老师、XX老师的评审及意见,答辩委员会老师聆听和指导。
        \end{enumerate}
      \end{block}
    }
    \frame{
      \frametitle{Q \& A}
      \begin{block}{Questions?}
       ~\\ ~\\
       \center{\Large{Thank you!}}
       \\ ~\\ ~\\ ~\\ ~\\ 
      \end{block}
    }

