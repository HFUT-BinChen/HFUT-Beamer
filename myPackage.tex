%导入一些用到的宏包
\usepackage{graphicx}
\usepackage{animate}
\usepackage{csvsimple}



% Chinese Package
\usepackage{xeCJK} %导入中文包
\setCJKmainfont{SimHei} %中文字体采用黑体
\setCJKsansfont{simhei.ttf} % for \textsf  Microsoft YaHei
%设置英文字体
%设置英文字体
\setsansfont{Microsoft YaHei}
\setmainfont{Microsoft YaHei}


%\setCJKmainfont[BoldFont={Adobe Heiti Std},ItalicFont={Adobe Kaiti Std}]{SimSun}


%\setCJKmainfont[ItalicFont={楷体_GB2312}, BoldFont={黑体}]{宋体} %衬线字体 缺省中文字体为
%\setCJKsansfont{黑体}        %serif是有衬线字体sans serif无衬线字体。
%\setCJKmonofont{仿宋_GB2312} %中文等宽字体
%%-----------------------xeCJK下设置中文字体------------------------------%
\setCJKfamilyfont{song}{SimSun}                             %宋体 song
\newcommand{\song}{\CJKfamily{song}}                        % 宋体   (Windows自带simsun.ttf)
%\setCJKfamilyfont{xs}{NSimSun}                              %新宋体 xs
%\newcommand{\xs}{\CJKfamily{xs}}
%\setCJKfamilyfont{fs}{FangSong_GB2312}                      %仿宋2312 fs
%\newcommand{\fs}{\CJKfamily{fs}}                            %仿宋体 (Windows自带simfs.ttf)
\setCJKfamilyfont{kai}{KaiTi_GB2312}                        %楷体2312  kai
\newcommand{\kai}{\CJKfamily{kai}}                          
\setCJKfamilyfont{yh}{Microsoft YaHei}                    %微软雅黑 yh
\newcommand{\yh}{\CJKfamily{yh}}
\setCJKfamilyfont{hei}{SimHei}                                    %黑体  hei
\newcommand{\hei}{\CJKfamily{hei}}                          % 黑体   (Windows自带simhei.ttf)
\setCJKfamilyfont{msunicode}{Arial Unicode MS}            %Arial Unicode MS: msunicode
\newcommand{\msunicode}{\CJKfamily{msunicode}}
\setCJKfamilyfont{li}{LiSu}                                            %隶书  li
\newcommand{\li}{\CJKfamily{li}}
\setCJKfamilyfont{yy}{YouYuan}                             %幼圆  yy
\newcommand{\yy}{\CJKfamily{yy}}
\setCJKfamilyfont{xm}{MingLiU}                                        %细明体  xm
\newcommand{\xm}{\CJKfamily{xm}}
\setCJKfamilyfont{xxm}{PMingLiU}                             %新细明体  xxm
\newcommand{\xxm}{\CJKfamily{xxm}}

\setCJKfamilyfont{hwsong}{STSong}                            %华文宋体  hwsong
\newcommand{\hwsong}{\CJKfamily{hwsong}}
\setCJKfamilyfont{hwzs}{STZhongsong}                        %华文中宋  hwzs
\newcommand{\hwzs}{\CJKfamily{hwzs}}
\setCJKfamilyfont{hwfs}{STFangsong}                            %华文仿宋  hwfs
\newcommand{\hwfs}{\CJKfamily{hwfs}}
\setCJKfamilyfont{hwxh}{STXihei}                                %华文细黑  hwxh
\newcommand{\hwxh}{\CJKfamily{hwxh}}
\setCJKfamilyfont{hwl}{STLiti}                                        %华文隶书  hwl
\newcommand{\hwl}{\CJKfamily{hwl}}
\setCJKfamilyfont{hwxw}{STXinwei}                                %华文新魏  hwxw
\newcommand{\hwxw}{\CJKfamily{hwxw}}
\setCJKfamilyfont{hwk}{STKaiti}                                    %华文楷体  hwk
\newcommand{\hwk}{\CJKfamily{hwk}}
\setCJKfamilyfont{hwxk}{STXingkai}                            %华文行楷  hwxk
\newcommand{\hwxk}{\CJKfamily{hwxk}}
\setCJKfamilyfont{hwcy}{STCaiyun}                                 %华文彩云 hwcy
\newcommand{\hwcy}{\CJKfamily{hwcy}}
\setCJKfamilyfont{hwhp}{STHupo}                                 %华文琥珀   hwhp
\newcommand{\hwhp}{\CJKfamily{hwhp}}

\setCJKfamilyfont{fzsong}{Simsun (Founder Extended)}     %方正宋体超大字符集   fzsong
\newcommand{\fzsong}{\CJKfamily{fzsong}}
\setCJKfamilyfont{fzyao}{FZYaoTi}                                    %方正姚体  fzy
\newcommand{\fzyao}{\CJKfamily{fzyao}}
\setCJKfamilyfont{fzshu}{FZShuTi}                                    %方正舒体 fzshu
\newcommand{\fzshu}{\CJKfamily{fzshu}}

\setCJKfamilyfont{asong}{Adobe Song Std}                        %Adobe 宋体  asong
\newcommand{\asong}{\CJKfamily{asong}}
\setCJKfamilyfont{ahei}{Adobe Heiti Std}                            %Adobe 黑体  ahei
\newcommand{\ahei}{\CJKfamily{ahei}}
\setCJKfamilyfont{akai}{Adobe Kaiti Std}                            %Adobe 楷体  akai
\newcommand{\akai}{\CJKfamily{akai}}

%------------------------------设置字体大小------------------------%
\newcommand{\chuhao}{\fontsize{42pt}{\baselineskip}\selectfont}     %初号
\newcommand{\xiaochuhao}{\fontsize{36pt}{\baselineskip}\selectfont} %小初号
\newcommand{\yihao}{\fontsize{28pt}{\baselineskip}\selectfont}      %一号
\newcommand{\erhao}{\fontsize{21pt}{\baselineskip}\selectfont}      %二号
\newcommand{\xiaoerhao}{\fontsize{18pt}{\baselineskip}\selectfont}  %小二号
\newcommand{\sanhao}{\fontsize{15.75pt}{\baselineskip}\selectfont}  %三号
\newcommand{\sihao}{\fontsize{14pt}{\baselineskip}\selectfont}%     四号
\newcommand{\xiaosihao}{\fontsize{12pt}{\baselineskip}\selectfont}  %小四号
\newcommand{\wuhao}{\fontsize{10.5pt}{\baselineskip}\selectfont}    %五号
\newcommand{\xiaowuhao}{\fontsize{9pt}{\baselineskip}\selectfont}   %小五号
\newcommand{\liuhao}{\fontsize{7.875pt}{\baselineskip}\selectfont}  %六号
\newcommand{\qihao}{\fontsize{5.25pt}{\baselineskip}\selectfont}    %七号
%------------------------------标题名称中文化-----------------------------%
\renewcommand\abstractname{\hei 摘\ 要}
\renewcommand\refname{\hei 参考文献}
\renewcommand\figurename{\hei 图}
\renewcommand\tablename{\hei 表}
%------------------------------定理名称中文化-----------------------------%
\newtheorem{dingyi}{\hei 定义~}[section]
\newtheorem{dingli}{\hei 定理~}[section]
\newtheorem{yinli}[dingli]{\hei 引理~}
\newtheorem{tuilun}[dingli]{\hei 推论~}
\newtheorem{mingti}[dingli]{\hei 命题~}
\newtheorem{lizi}{{例}}


%算法设置
\usepackage{algorithm}  
\usepackage{algorithmicx}  
\usepackage{algpseudocode}
\floatname{algorithm}{算法}
\renewcommand{\algorithmicrequire}{\textbf{输入:}} 
\renewcommand{\algorithmicensure}{\textbf{输出:}}  

\algrenewcommand{\algorithmiccomment}[1]{ $//$ #1}
%代码设置
\usepackage{fancybox}
\usepackage{xcolor}
\usepackage{times}
\usepackage{listings}

\renewcommand\lstlistingname{代码}
\renewcommand\lstlistlistingname{代码}

\lstset{framexleftmargin=1.4em,
        xleftmargin=1.8em,
        basicstyle=\ttfamily\small,
        %frame=shadowbox, numberstyle=\tiny, breaklines=true,
        frame=single,
        numberstyle=\tiny, breaklines=true,
        keywordstyle=\color{blue!70}\bfseries,
        %commentstyle=\color{red!50!green!50!blue!50},
        rulesepcolor=\color{red!20!green!20!blue!20},
        numbers=none,fontadjust=true}
\lstdefinelanguage{shader}{morekeywords={uniform, layout, uniform, vec2, vec3, vec4, in, out, gl_Position, dot, flat, int ,float, gl_VertexID, xyz, w, x, y, z, location, version, sampler2DRect, bgr, gl_FragData, texture2DRect, gl_TexCoord,for,xy},morecomment=[l]{//}}



\usepackage{latexsym,amssymb,amsmath,amsbsy,amsopn,amstext,xcolor,multicol}
\usepackage{graphicx,wrapfig,fancybox}
\usepackage{pgf,pgfarrows,pgfnodes,pgfautomata,pgfheaps,pgfshade}

%\usepackage[backend=bibtex,style=IEEE,sorting=none]{biblatex} % [参考文献格式](https://www.sharelatex.com/blog/2013/07/31/getting-started-with-biblatex.html)
\usepackage[backend=bibtex,sorting=none]{biblatex} % [参考文献格式](https://www.sharelatex.com/blog/2013/07/31/getting-started-with-biblatex.html) %mac IEEE not found
%\usepackage{array}
\usepackage{bm}
\usepackage{caption}
\usepackage[caption=false,font=scriptsize]{subfig}
\usepackage{multirow}
\usepackage{booktabs}



\defbibheading{bibliography}[\bibname]{} %avoid printbibliography 自动生成目录
\addbibresource{ref/papers-bib-in-google.bib}
\addbibresource{ref/chinese-ref.bib}
%\setbeamertemplate{bibliography item}{\insertbiblabel} %将列表中默认的丑陋的icon 改成数字,或者下面这个也行
\setbeamertemplate{bibliography item}[text] % [ref](http://tex.stackexchange.com/questions/68080/beamer-bibliography-icon)
%\setbeamertemplate{footline}[frame number]{}

%\setframeofframes{of}

\usepackage{boxedminipage} %for: bvh border
\def\fourgraphicswidth{0.35} %0.3\textwidth



